%************************************************************************
%
%  Project: Cerberus
%  Title: Cerberus Design Document
%  Author: Marc Streit, Michael Kalkusch
%  Advisor:
%  
%  Filneame: main.tex
%  Contents: main file, LaTeX layouts, includes
%
%  Original version: 24.07.2006
%  Original author: Marc Streit
%  Supporting author:
%  Version number: 1
%  Last change: 24.07.2006
%  Last change by: Marc Streit
%
%************************************************************************

%************************************************************************
%
%  !!! Important NOTE !!! To use these templates (including this file and
%  the main.tex file) you have to copy the folder custom_styles to your texmf
%  and then execute "mktexlsr ~/texmf/"
%
%************************************************************************

\documentclass[a4paper, 12pt, BCOR5mm, footsepline, parskip+, chapterprefix, smallheadings, footexclude, headexclude, DIV14, abstracton, pointlessnumbers, tablecaptionabove]{scrreprt}

%===========================================================
% 
\usepackage[english, german]{babel}
\usepackage[T1]{fontenc}
\usepackage{ae,aecompl}
\usepackage[latin1]{inputenc}
%===========================================================

%===========================================================
% including graphical packages
\usepackage[pdftex]{graphicx}
\usepackage{wrapfig}
%===========================================================

% Hyperrefs
\usepackage{url}
%\usepackage[hyperindex, plainpages=false, pdfpagelabels]{hyperref}
\usepackage[]{dino_references}

% Bibliographic Package
\usepackage{natbib}

% Packages for Source Code
\usepackage{listings}
\lstset{
  language=Matlab,
  showstringspaces=false,
  numbers=left,
  frame=single
  breaklines=true
}

% set how many hirachies should be in the table of contents
\setcounter{tocdepth}{3}

% set how many hirachies should be numbered 
\setcounter{secnumdepth}{4}

%===========================================================
% page layout

\usepackage{scrpage2}
\pagestyle{scrheadings}
\renewcommand*{\chapterpagestyle}{scrheadings}
\renewcommand*{\partpagestyle}{empty}
\renewcommand{\headfont}{\normalfont\sffamily\bfseries}
\renewcommand{\pnumfont}{\normalfont\sffamily\bfseries}
\automark{chapter}
\cfoot{}
\chead{}
%\lefoot{\pagemark}
\ofoot{\headmark \hspace{1em}|\hspace{1em}\pagemark}
\ohead{}

\usepackage{lscape}
%===========================================================

%===========================================================
% paragrgaph layout
\setlength{\parskip}{1ex} % distance between paragraphs

% distance between lines
%\usepackage{setspace}
%\doublespacing

%===========================================================
% table formatting
%\usepackage{longtable} % for long tables (more than a page)
%\usepackage{capt-of}   % for 
%\usepackage{rccol}
%===========================================================


%===========================================================
% Author Data
\subject{Design Document \\ Institute for Computer Graphics and Vision \\ Graz University of Technology \\ Graz}
\title{Cerberus\\ \large Design Document}
\author{Marc Streit \\ marc.streit@student.tugraz.at \and  DI Michael Kalkusch \\ kalkusch@icg.tugraz.at}
\date{July 2006}
%\publishers{Supervision: }
%===========================================================

%===========================================================
\begin{document}

% for english
\selectlanguage{english}
% for german
%\selectlanguage{german}
%===========================================================
\maketitle

%\begin{abstract}

\end{abstract}

\selectlanguage{english}

I hereby certify that the work presented in this diploma thesis is my own and that work performed by others is appropriately cited.

\selectlanguage{german}

Ich versichere hiermit, diese Arbeit selbst�ndig verfa�t, keine anderen als die angegebenen Quellen und Hilfsmittel benutzt und mich auch sonst keiner unerlaubten Hilfsmittel bedient zu haben.


%%% Local Variables: 
%%% mode: latex
%%% TeX-master: "main"
%%% End: 
            % Abstracts, Pledge
 
%\minisec{Acknowledgements}





%%% Local Variables: 
%%% mode: plain-tex
%%% TeX-master: "main"
%%% End: 


\tableofcontents             % Table of Contents

%--chapters-------------------

\chapter{Introduction}

\chapter{Problem Definiton}

%Some types of cancer are suspected to either originate in or else being influenced by genetic disposition. Though human DNA has been examined and broken down to sequences of genes the meaning of most genes and their influence on each other is still unknown. The Cerberus application enables doctors to get an overview over patients' gene-expression data as well as clinical data like the history of the disease. It uses clustering techniques to provide information visualization, thereby allowing doctors to systematically search for coincidences. The aim of this thesis is to extend the application with a metabolic pathway module to take the connection between metabolic pathways and genes into account. 

\chapter{Related Work}

\section{Biomedical Databases}

%INSERT table with current data sets in various databases

%Describe criteria of ``good'' database. 

\subsection{Entrez}

%INSERT figure of the entrez network.

\subsection{Kyoto Encyclopedia of Genes and Genomes (KEGG)}

%The Kyoto Encyclopedia of Genes and Genomes (KEGG)\footnote{http://www.genome.jp/kegg/} is a biomedical resource that started its %online service in 1995 and belongs to the Japanese GenomeNet.

\subsection{Gene Ontology (GO)}

\subsection{Biomedical Identification Numbers}

\section{Metabolic Pahtway Visualization}

\subsection{Medical Pathways}
\subsection{State-of-the-art Frameworks}

\section{Gene Expression Visualization}

\subsection{Gene-Expression Analysis}
\subsection{State-of-the-art Frameworks}

\section{Information Visualization Methods}

\subsection{Multiple Views}
\subsection{Focus + Context}
\subsection{Linking \& Brushing}
\subsection{Semantic Zooom}
\subsection{Neighborhood Visualization}

\section{Application of Gene-Expression Data onto Metabolic Pathways}

\section{Pathway Visualization influencing Gene-Expression Analysis}

\chapter{System Architecture}

\section{Overall Design}
\section{Data Management}
\subsection{Pathway Data}
\subsection{Set / Storage / Virtual Array}

%Referenz auf Michaels Thesis

\section{Graphical User Interface (GUI)}

\section{Data Update Mechanism}

\section{Selection Handling}

\chapter{Implementation}

\section{Used Technologies}
\subsection{SWT}

%SWT successor of AWT.
%Abstract Widget Toolkit (AWT)

\subsection{JOGL}

%Java OpenGL library\footnote{https://jogl.dev.java.net/}

%\subsection{OpenGL integration in Java}
\subsection{JGraph}

\section{Data Loading}

\section{Visualization Techniques}

\subsection{2D Pathway Implementation}
\subsubsection{Pathway Switching}
\subsubsection{Hierarchical Pathways}
\subsubsection{Neighborhood Visualization}

\subsection{OpenGL Pathways}
\subsubsection{Pathway Texture Overlay}
\subsubsection{Pathway Linking}
\subsubsection{Pathway Element Picking}
\subsubsection{Hierarchical Display Lists}
\subsubsection{Layered Pathways}
\subsubsection{Panel View}

\chapter{Results}

\chapter{Conclusions}

\chapter{Future Work}

\index{bla}



%\include{demo}

\appendix

\listoffigures

%\listoftables

%use only if you want to list all!!! entrys in your bibliography
\nocite{*} 
\bibliography{biblio/biblio}
% for german use this:
%\bibliographystyle{/usr/share/texmf/custom_styles/alphadin}
% for english this:
\bibliographystyle{/usr/share/texmf/custom_styles/alphadin_en}

%=========================================================== 
 

%===========================================================
\end{document}
%===========================================================

%%% Local Variables: 
%%% mode: pdflatex
%%% TeX-master: t
%%% End: 